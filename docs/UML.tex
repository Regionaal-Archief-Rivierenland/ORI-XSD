\documentclass[margin=10pt]{standalone}

\usepackage{tikz}
\usepackage{amsmath}
\usepackage{mathtools}
\usepackage[kerning=true, tracking=true]{microtype}
\usepackage[dutch]{babel}

\usepackage{inconsolata}
%% computer modern monospace?
%% \usepackage[tt={oldstyle=false,variable=false}]{cfr-lm}

\usepackage[T1]{fontenc}

\usepackage{listings}
\usepackage{minted}
\usemintedstyle{vs}

%% Color definitions
\usepackage[dvipsnames]{xcolor}
\definecolor{githubgrey}{HTML}{EAEEF2}
\definecolor{darkgreen}{HTML}{3D5E45}
\definecolor{purple}{HTML}{5C5573}
%% rounded boxes
\usepackage[most]{tcolorbox}

%% for templates
\usepackage{xparse}
\newcommand\mystrut{\rule[-2.5pt]{0pt}{12pt}}
%% datatype typesetting
\newcommand{\datatype}[2][text]{%
    \tcbox[
      on line,
      boxsep=0pt,
      left=2.5pt,
      right=2.5pt,
      top=0pt,
      bottom=0pt,
      enlarge top initially by=-14pt,
      enlarge bottom by=-14pt,
      enlarge right by=-4pt,
      enlarge left by=-2pt,
      opacityframe=0,
      colback=githubgrey,
      fontupper={\ttfamily\mystrut},
      fontlower={\ttfamily\mystrut}]
      {\textcolor{black}{\textbf{#2}}}
}

\newcommand{\datakeuze}[2][text]{%
    \tcbox[
      on line,
      boxsep=0pt,
      left=2.5pt,
      right=2.5pt,
      top=0pt,
      bottom=0pt,
      enlarge top initially by=-14pt,
      enlarge bottom by=-14pt,
      enlarge right by=-4pt,
      enlarge left by=-2pt,
      opacityframe=0,
      colback=githubgrey,
      fontupper={\ttfamily\mystrut},
      fontlower={\ttfamily\mystrut}]
      {\textcolor{black}{\textbf{#2}}}
}

\usepackage[]{MyriadPro}

\begin{document}

%% set default font to myriad pro
%% \renewcommand{\familydefault}{\sfdefault}
\sffamily

\usetikzlibrary{arrows}
\usetikzlibrary{shapes.multipart}
\usetikzlibrary{positioning}

%% TODO
%% - check if different number style in Myriad pro looks nice
%% - try semi boldfont for Attribute names
%%   - too difficult with pdflatex
%% - maybe bullets instead of plusses? Making a proper list also gives more flexibility
%%    - all the more because the plusses serve little purposes


%% Header of boxes
\newcommand{\objectheader}[1]{{\large \textbf{#1}}
  \nodepart{two}
}
%% Attribute typesetting
\newcommand{\attribute}[3]{+ \; #1: \datatype{#2} #3}

\begin{tikzpicture}[
    %% not sure why the "header" is still centered after align=left, but oh well
    ORI object/.style={rectangle split, rectangle split parts=2,
      draw=black, align=left, rounded corners=5pt, very thick, inner sep=10pt}]
  
  \node (agendapunt) [ORI object]
        {
          \objectheader{Objecttype::Agendapunt}

          \attribute{ID}{string} \\
          \attribute{Agendapunt omschrijving}{string}{[0..1]} \\
          \attribute{Gepland agendapuntvolgnummer}{string}{[0..1]} \\
          \attribute{Agendapuntvolgnummer}{string}{[0..1]} \\
          \attribute{Geplande starttijd}{datum + tijd}{[0..1]} \\
          \attribute{Geplande eindtijd}{datum + tijd}{[0..1]} \\
          \attribute{Starttijd}{datum + tijd}{[0..1]} \\
          \attribute{Eindtijd}{datum + tijd}{[0..1]} \\
          \attribute{Agendapunt kenmerk}{string}{[0..1]} \\
          \attribute{Agendapunt titel}{string}{[0..1]} \\
          \attribute{Indicatie hamerstuk}{boolean}{[0..1]} \\
          \attribute{Indicator behandeld}{boolean}{[0..1]} \\
          \attribute{Indicator besloten}{boolean}{} \\
          \attribute{Bestuurslaag}{bestuurslaagGegevens}{[0..1]}
        };

  \node (stemming) [ORI object, below = 1cm of agendapunt]
        {
          \objectheader{Objecttype::stemming}

          \attribute{ID}{string} \\
          \attribute{StemmingsType}{Hoofdelijk | Regulier | Schriftelijk} \\
          \attribute{Resultaat mondelinge stemming}{Voor | Tegen | Gelijk} \\
          \attribute{Resultaat stemming over personen}{string} \\
          \attribute{Stemming over personen}{stemmingOverPersonenGegevens} \\
        };
\end{tikzpicture}

\end{document}
